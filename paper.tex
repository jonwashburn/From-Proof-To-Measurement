\documentclass[11pt]{article}
\usepackage{amsmath,amssymb,amsfonts}
\usepackage[hidelinks]{hyperref}
\usepackage{verbatim}
\usepackage{amsthm}
\allowdisplaybreaks
\theoremstyle{plain}
\newtheorem{theorem}{Theorem}
\newtheorem{lemma}{Lemma}
\newtheorem{proposition}{Proposition}
\newtheorem{corollary}{Corollary}
\theoremstyle{definition}
\newtheorem{definition}{Definition}
\theoremstyle{remark}
\newtheorem*{remark}{Remark}

\title{From Proof to Measurement:\\
A Reality Bridge with Falsifiable SI Landings}
\author{Jonathan Washburn\\
Recognition Science, Recognition Physics Institute\\
Austin, Texas, USA\\
\texttt{washburn.jonathan@gmail.com}}
\date{} % leave blank per journal style

\begin{document}
\maketitle

\begin{abstract}
This methods paper defines a proof–verified semantics that carries a strictly dimensionless derivation layer into SI displays without introducing tunable parameters. The inputs (proved upstream and not re-proved here) are: a unique symmetric multiplicative cost \(J(x)=\tfrac12(x+x^{-1})-1\) with log-axis form \(J(e^{t})=\cosh t-1\); a quantized tick with exact \(n\cdot\delta\) increments and gauge fixed up to a componentwise constant; an eight-phase minimal cycle in three-bit parity space; and the golden-ratio gap \(\ln\varphi\). We formalize a \emph{Reality Bridge} that (i) displays \(J\) additively as \(S/\hbar\) (no offset, no fit), (ii) assigns a recognition tick \(\tau_{\mathrm{rec}}=(2\pi/(8\ln\varphi))\,\tau_{0}\) and a kinematic hop length \(\lambda_{\mathrm{kin}}=c\,\tau_{\mathrm{rec}}\) with \(c=\ell_{0}/\tau_{0}\), and (iii) provides two independent SI landings (time-first and length-first) whose numerical agreement within stated combined uncertainty constitutes a single pass/\,fail criterion. We prove non-circularity (unit relabelings factor out and cannot alter dimensionless content) and uniqueness at the stated symmetry (the bridge is fixed up to trivial unit rescalings). No sector models, regressions, thresholds, or empirical tuning are used. A reproducibility pack (Lean theorem identifiers, checksums, and one-command scripts that compute both landings and the pass/\,fail statistic) is specified for audit.
\end{abstract}

\section{Introduction}
\paragraph{Problem.}
Mathematical results are exact and dimensionless; measurements are finite-precision and SI-native. Claims of being “parameter-free” often collapse under audit because units and calibrations quietly inject knobs. The challenge is to expose a route from theorem to instrument readout that is (a) explicit, (b) auditable, and (c) falsifiable—without feeding any parameter back into the proofs.

\paragraph{Starting point (dimensionless inputs).}
We assume, as upstream facts proved elsewhere and not re-proved here:
\begin{itemize}
  \item a unique symmetric multiplicative cost \(J(x)=\tfrac12(x+x^{-1})-1\) with log-axis minimum at \(x=1\);
  \item a quantized tick on a discrete ledger so that \(n\) steps produce an exact increment \(n\cdot\delta\);
  \item an eight-tick partition in three-bit parity space (minimal period \(8\));
  \item the golden ratio \(\varphi=\frac{1+\sqrt{5}}{2}\) as the positive solution of \(x=1+\frac1x\), with gap \(\delta_{\mathrm{gap}}=\ln\varphi\).
\end{itemize}
These are purely dimensionless. They are the only features the measurement layer is allowed to see.

\paragraph{Reality Bridge (what we introduce).}
We define a single semantics that:
\begin{enumerate}
  \item displays \(J\) additively as action via \(J\mapsto S/\hbar\) (a naming, not a fit);
  \item assigns the ledger tick an SI duration
\begin{align*}
 \left|\frac{\lambda_{\mathrm{kin}}-\lambda_{\mathrm{rec}}}{\lambda_{\mathrm{rec}}}\right| &\le k\,u_{\mathrm{comb}},\\
 u_{\mathrm{comb}} &= \sqrt{u(\lambda_{\mathrm{kin}})^{2}+u(\lambda_{\mathrm{rec}})^{2}-2\rho\,u(\lambda_{\mathrm{kin}})\,u(\lambda_{\mathrm{rec}})}.
\end{align*}

  \item \emph{Conventional anchor \(\lambda_{\mathrm{rec}}\) (Route B).} Adopt
  \[
  \lambda_{\mathrm{rec}}:=\sqrt{\frac{\hbar\,G}{c^{3}}}.
  \]
  In SI, \(c\) and \(h\) (hence \(\hbar=h/2\pi\)) are exact; the relative uncertainty is dominated by \(G\). \textbf{Frozen for this submission:}
  \[
  u(G)=2.0\times10^{-5}\quad\Longrightarrow\quad u(\lambda_{\mathrm{rec}})=\tfrac12\,u(G)=1.0\times10^{-5}.
  \]
  \item \emph{Correlation between landings.} Use disjoint laboratories (or, at minimum, disjoint hardware and analysis chains) for Routes A and B. \textbf{Chosen value:}
  \[
  \rho=0\quad\text{(engineered independence)}.
  \]
\end{enumerate}

\textbf{Implied combined uncertainty.} With the above targets and \(\rho=0\),
\[
u_{\mathrm{comb}}\,=\,\sqrt{\,u(\ell_{0})^{2}+u(\lambda_{\mathrm{rec}})^{2}\,}\approx 1.0\times10^{-5}.
\]

\textbf{Audit note.} These values are \emph{pre\-declared}. They may be updated only by issuing a new artifact pack (e.g., if a revised recommended value of \(G\) is adopted); retroactive changes after observing the comparison are not permitted.
\end{remark}



\section{Operational Protocols (How to Measure Without Knobs)}

\subsection*{Protocol A: Clock-side Determination of \(\tau_{0}\)}
\textbf{Objective.} Realize the SI second (time unit) and compute the recognition tick and kinematic hop length without introducing tunable parameters.

\textbf{Instruments.} One of:
(i) in-lab primary/secondary time standard (e.g., a cesium fountain or an optically steered hydrogen maser with a frequency comb), or
(ii) a calibrated UTC(k) realization with traceable time-transfer (e.g., common-view GNSS or two-way satellite/optical fiber links).

\textbf{Procedure.}
\begin{enumerate}
  \item \emph{Realize the second.} Lock a local oscillator to the SI definition of the second. Record the relative standard uncertainty \(u(\tau_{0})\) from the comparison interval and reported stability (Allan deviation\cite{Allan1966,NISTSP1065}) of the realization.\footnote{Display-level identities treat \(\tau_{0}\) as a unit name; lab realizations carry finite \(u(\tau_{0})\).}
  \item \emph{Compute the recognition tick.} Set
  \[
  \tau_{\mathrm{rec}}=\frac{2\pi}{8\ln\varphi}\,\tau_{0}.
  \]
  This identity has no fit parameter and inherits the relative uncertainty of \(\tau_{0}\) at realization level.
  \item \emph{Compute the kinematic hop length.} With \(c:=\ell_{0}/\tau_{0}\),
  \[
  \lambda_{\mathrm{kin}}=c\,\tau_{\mathrm{rec}}=\frac{2\pi}{8\ln\varphi}\,\ell_{0}.
  \]
  Note the cancellation: realization noise in \(\tau_{0}\) cancels algebraically; the display depends only on the length unit name \(\ell_{0}\). If \(\ell_{0}\) is treated as a unit name (no physical realization in this step), then \(u(\lambda_{\mathrm{kin}})=0\) at the display level; if a physical length realization is used, set \(u(\lambda_{\mathrm{kin}})=u(\ell_{0})\).
  \item \emph{Record invariants.} Report the two normalized, unit-invariant ratios
  \[
  \frac{\tau_{\mathrm{rec}}}{\tau_{0}}=\frac{2\pi}{8\ln\varphi},\qquad
  \frac{\lambda_{\mathrm{kin}}}{\ell_{0}}=\frac{2\pi}{8\ln\varphi}.
  \]
\end{enumerate}

\textbf{Targets.} Aim for \(u(\tau_{0})\le 10^{-15}\) (clock realization over multi-hour averaging) and, if a physical length realization is invoked, \(u(\ell_{0})\le 10^{-9}\). These targets are illustrative and may be tightened by the laboratory.

\textbf{Acceptance.} No fitting or thresholding is performed. The outputs are the identities above; uncertainty is documented, not tuned.

\medskip

\subsection*{Protocol B: Length-side Determination of \(\lambda_{\mathrm{rec}}\)}
\textbf{Objective.} Land independently on a conventional hop-length anchor and infer the same \(\tau_{\mathrm{rec}}\) through kinematics.

\textbf{Anchor choice.} Adopt the conventional definition
\[
\lambda_{\mathrm{rec}}:=\sqrt{\frac{\hbar\,G}{c^{3}}}.
\]
Here \(c\) and \(\hbar\) are exact in SI; \(G\) carries the relative standard uncertainty \(u(G)\). Consequently,
\[
u(\lambda_{\mathrm{rec}})=\tfrac12\,u(G).
\]

\textbf{Independence.} Realize \(\lambda_{\mathrm{rec}}\) using a calibration and analysis chain that is organizationally and instrumentally disjoint from Protocol A (different laboratory or at minimum a distinct hardware chain and data reduction), so the correlation coefficient \(\rho\) between the relative estimates of \(\ell_{0}\) (from Protocol A, if realized) and \(\lambda_{\mathrm{rec}}\) is engineered to be near zero.

\textbf{Procedure.}
\begin{enumerate}
  \item \emph{Evaluate the anchor.} Compute \(\lambda_{\mathrm{rec}}\) from the adopted constants and document \(u(\lambda_{\mathrm{rec}})=\tfrac12 u(G)\).
  \item \emph{Infer the recognition tick.} With the exact identity \(c=\ell_{0}/\tau_{0}\),
  \[
  \tau_{\mathrm{rec}}=\frac{\lambda_{\mathrm{rec}}}{c}=\lambda_{\mathrm{rec}}\ \frac{\tau_{0}}{\ell_{0}}.
  \]
  This step is a display conversion; no fit is introduced.
  \item \emph{Report invariants.} Verify that
  \[
  \frac{\tau_{\mathrm{rec}}}{\tau_{0}}=\frac{\lambda_{\mathrm{rec}}}{\ell_{0}},\qquad
  \frac{\lambda_{\mathrm{rec}}}{\ell_{0}}=\frac{2\pi}{8\ln\varphi}
  \]
  are numerically consistent with Protocol A within the uncertainty model specified in the previous section.
\end{enumerate}

\textbf{Targets.} Use the current recommended value of \(G\) with its stated standard uncertainty. No additional parameters are introduced.

\medskip

\subsection*{Protocol C: Cross-sector Consistency}
\textbf{Objective.} Check that the action display \(S/\hbar\) corresponding to a given dimensionless stretch/compression \(x\) (hence a fixed \(J(x)\)) is invariant across distinct experimental contexts.

\textbf{Contexts.} Perform at least two of the following, ensuring independent instrumentation and analysis:
\begin{enumerate}
  \item \emph{Ramsey phase accumulation (two-level system).} Implement a controlled detuning \(\Delta\) for time \(T\). The accumulated phase \(\Phi=\Delta T\) gives a dimensionless action display via \(S/\hbar=\Phi\). Choose \(\Delta\) and \(T\) to realize a target \(J\) on the log axis, \(J(e^{t})=\cosh t-1\), and compare the inferred \(S/\hbar\) to \(J\).
  \item \emph{Josephson phase evolution.} In a current-biased Josephson junction, the phase obeys \(\dot{\phi}=(2e/\hbar)V\). Integrate over a controlled voltage step to obtain \(\Delta\phi\) and report \(S/\hbar=\Delta\phi\). Match to the same target \(J\).
  \item \emph{Optical cavity stretch.} Modulate cavity length by a calibrated fractional change \(x=e^{t}\). The mode frequency scales multiplicatively; the log-axis form \(J(e^{t})=\cosh t-1\) allows a direct comparison of the measured action-equivalent phase advance (via frequency–time area) to \(J\).
\end{enumerate}

\textbf{Procedure.}
\begin{enumerate}
  \item \emph{Select a target \(t\)} (e.g., \(t=\pm 0.01\)) and compute \(J(e^{t})=\cosh t-1\).
  \item \emph{Configure each context} to produce the same \(t\) (within control tolerances). Extract an experimental estimate of \(S/\hbar\) from each context as described above.
  \item \emph{Compare displays.} For any two contexts \(A,B\), form
  \[
  \Delta_{AB}:=\left|\frac{(S/\hbar)_A-(S/\hbar)_B}{J(e^{t})}\right|.
  \]
  \item \emph{Acceptance.} Require \(\Delta_{AB}\le k\,u_{AB}\), where \(u_{AB}\) is the combined relative standard uncertainty for the pair (including any declared correlation) and \(k\in\{1,2\}\) is the coverage factor. Report all \(\Delta_{AB}\) and \(u_{AB}\).
\end{enumerate}

\textbf{Notes.}
(i) No regression, thresholds, or empirical tuning are permitted; each context maps directly to a display \(S/\hbar\) that must coincide with the mathematical \(J\) at the chosen \(t\).
(ii) Control \(t\) symmetrically (\(+t\) and \(-t\)) to expose even/odd systematics; \(J\) is even in \(t\).
(iii) Keep hardware and analysis chains disjoint across contexts to minimize cross-correlation.

\section{No-Knob Accounting (Why This is Parameter-Free)}

\paragraph{Definitions.}
A \emph{knob} is any adjustable numerical parameter chosen or tuned to improve agreement with data (including implicit choices such as ex–post coverage, regression weights, or selective averaging). A \emph{unit label} is a naming pair \((\tau_{0},\ell_{0})\) for seconds and meters. Unit labels may be \emph{realized} with finite uncertainty, but algebraically they act as symbols that must cancel in normalized displays.

\paragraph{Derivation layer (purely dimensionless).}
The only inputs used by the bridge are theorem-level equalities that carry no units:
\[
\frac{\tau_{\mathrm{rec}}}{\tau_{0}}=\frac{2\pi}{8\ln\varphi},\qquad
\frac{\lambda_{\mathrm{kin}}}{\ell_{0}}=\frac{2\pi}{8\ln\varphi},\qquad
\frac{S}{\hbar}=J.
\]
These relations are fixed by proof and cannot be altered by any laboratory choice.

\paragraph{Display layer (names, not fits).}
The bridge \emph{names} SI displays via
\[
\tau_{\mathrm{rec}}=\frac{2\pi}{8\ln\varphi}\,\tau_{0},\qquad
\lambda_{\mathrm{kin}}=c\,\tau_{\mathrm{rec}}=\frac{2\pi}{8\ln\varphi}\,\ell_{0},\qquad
c=\frac{\ell_{0}}{\tau_{0}}.
\]
No regression, priors, thresholds, or free coefficients appear. The only variability is metrological uncertainty in the \emph{realization} of unit labels or anchors, which is documented—not tuned.

\begin{lemma}[Knob nullity]
Let \(\theta\) denote any continuous adjustment at the display level (choice of weights, offsets, or fit parameters). Then
\[
\frac{\partial}{\partial\theta}\left(\frac{\tau_{\mathrm{rec}}}{\tau_{0}}\right)
=
\frac{\partial}{\partial\theta}\left(\frac{\lambda_{\mathrm{kin}}}{\ell_{0}}\right)
=
\frac{\partial}{\partial\theta}\left(\frac{S}{\hbar}\right)
=0.
\]
\end{lemma}

\begin{proof}
Each normalized quantity equals a fixed constant or a dimensionless theorem (\(2\pi/(8\ln\varphi)\) or \(J\)). By algebra, unit relabelings \((\tau_{0},\ell_{0})\mapsto(\alpha\tau_{0},\beta\ell_{0})\) cancel in the ratios; any additional display-level adjustment \(\theta\) is external to the identities and cannot enter these expressions.
\end{proof}

\paragraph{Falsifiability rule (single inequality).}
All experimental checks reduce to one auditable comparison with pre\-declared coverage \(k\in\{1,2\}\):
\[
\left|\frac{\lambda_{\mathrm{kin}}-\lambda_{\mathrm{rec}}}{\lambda_{\mathrm{rec}}}\right|
\ \le\ 
k\,u_{\mathrm{comb}},
\qquad
u_{\mathrm{comb}}\,=\,\sqrt{\,u(\lambda_{\mathrm{kin}})^{2}+u(\lambda_{\mathrm{rec}})^{2}-2\rho\,u(\lambda_{\mathrm{kin}})\,u(\lambda_{\mathrm{rec}})\,}.
\]
No parameter may be adjusted after seeing the data;
\noindent As a computational alternative for non-linear propagation, the Monte Carlo supplement may be used~\cite{JCGM101}.
 \(\rho\) (correlation) and \(k\) are declared \emph{a priori}.

\paragraph{What counts as a knob (forbidden).}
(i) altering \(k\) after observing the discrepancy; (ii) reweighting or trimming data ex–post to reduce the discrepancy; (iii) introducing offsets/scales in \(S/\hbar\) (would violate \(J(1)=0\)); (iv) redefining anchors post hoc.

\paragraph{What does \emph{not} count as a knob (allowed).}
(i) choosing \((\tau_{0},\ell_{0})\) as unit names; (ii) reporting certified \(u(\cdot)\) from traceable calibrations~\cite{ISO17025,CIPM_MRA}; (iii) declaring \(\rho\) based on design (disjoint chains) or using a conservative bound; (iv) re-running the same fixed pipeline on new data.

\paragraph{Consequence.}
Because the derivation layer is dimensionless and the display layer is algebraic, there is no free parameter capable of changing the decision outcome except by changing declared uncertainty or correlation—both of which must be specified before observing the comparison. Hence the program is parameter-free in the operational sense required for audit.

\section{What Success or Failure Would Mean}

\subsection*{If the inequality holds (within stated \(k\))}
\textbf{Interpretation.} The Reality Bridge is \emph{operationally consistent} at the tested precision: the two independently realized SI landings agree within the pre\-declared coverage \(k\) and combined relative uncertainty \(u_{\mathrm{comb}}\). No parameters have been tuned to obtain this result.

\textbf{Immediate consequences.}
\begin{itemize}
  \item The bridge invariants are empirically supported:
  \[
  \frac{\tau_{\mathrm{rec}}}{\tau_{0}}=\frac{2\pi}{8\ln\varphi},\qquad
  \frac{\lambda_{\mathrm{kin}}}{\ell_{0}}=\frac{2\pi}{8\ln\varphi},\qquad
  \frac{S}{\hbar}=J.
  \]
  \item Proceed to sector-specific applications \emph{without} feeding data back into proofs: use the same fixed semantics and uncertainty policy.
  \item Replicate with independent hardware/teams to check stability of the pass/\,fail statistic.
  \item Tighten uncertainty budgets (smaller \(u(\ell_{0})\), independent \(\lambda_{\mathrm{rec}}\)) if higher precision is desired.
\end{itemize}

\subsection*{If it fails (persistently, after controls)}
\textbf{Interpretation.} Either (i) the Reality Bridge semantics is wrong for the stated mapping, or (ii) at least one landing assumption (anchor, traceability, independence, or declared correlation) is invalid. For the present semantics, the program is \emph{falsified}.

\textbf{Controls (must be verified before declaring failure).}
\begin{itemize}
  \item Re-run the \emph{fixed} pipeline end-to-end on fresh data (no code or parameter changes).
  \item Confirm that Route A and Route B use disjoint calibration chains and that the declared correlation \(\rho\) is accurate or conservatively bounded.
  \item Re-check that the adopted constants and unit labels are exactly those stated (no quiet revisions of anchors or values).
  \item Verify that no thresholds, offsets, weights, or regressions were introduced post hoc.
\end{itemize}

\textbf{Next actions after confirmed failure.}
\begin{itemize}
  \item Publish the negative result with full artifact trail (scripts, hashes, uncertainty accounting, and raw data).
  \item If a new semantics or a different landing is proposed, treat it as a \emph{new hypothesis}: restate claims, predeclare anchors/coverage/correlation, and rerun the same single-inequality test. Retroactive edits to make the present test pass are not permitted.
\end{itemize}

\section{Artifact and Audit Trail}

\paragraph{Purpose.}
This section fixes the files, identifiers, and deterministic steps required to audit the paper’s claims. No networks or external resources are needed; all computations are reproducible from the artifact set alone.

\subsection*{A. Artifact set (file-by-file)}
\begin{itemize}
  \item \texttt{paper.tex} — the canonical source of the manuscript.
  \item \texttt{paper.pdf} — the rendered manuscript corresponding to \texttt{paper.tex}.
  \item \texttt{manifest.txt} — human-readable list of all files in this bundle with size (bytes) and SHA-256.
  \item \texttt{versions.txt} — exact toolchain versions and environment variables used for deterministic builds.
  \item \texttt{invariants.txt} — the three bridge invariants printed symbolically:
\begin{align*}
 c &\mapsto c' = \frac{\ell_{0}'}{\tau_{0}'} = \frac{\beta}{\alpha}\,c, \\ 
 \tau_{\mathrm{rec}} &\mapsto \tau_{\mathrm{rec}}' = K\,\tau_{0}' = \alpha\,\tau_{\mathrm{rec}}, \\ 
 \lambda_{\mathrm{kin}} &\mapsto \lambda_{\mathrm{kin}}' = c'\tau_{\mathrm{rec}}' = \beta\,\lambda_{\mathrm{kin}}.
\end{align*}

  with \(J(1)=0\) and \(J(x)=J(1/x)\).
  \item \textbf{Phase per full cycle:} one complete cycle corresponds to a phase advance of \(2\pi\).
  \item \textbf{Ticks per cycle:} in three-bit parity (\(D=3\)), the minimal cycle length is \(M=2^{D}=8\) ticks.
  \item \textbf{Golden-ratio gap:} \(\varphi=\tfrac{1+\sqrt5}{2}\), \(\delta_{\mathrm{gap}}=\ln\varphi\).
  \item \textbf{Recognition tick and hop length:}
  \[
  K:=\frac{2\pi}{8\ln\varphi},\qquad
  \tau_{\mathrm{rec}}=K\,\tau_{0},\qquad
  c=\frac{\ell_{0}}{\tau_{0}},\qquad
  \lambda_{\mathrm{kin}}=c\,\tau_{\mathrm{rec}}=K\,\ell_{0}.
  \]
  \item \textbf{Action display:} \(S/\hbar=J\).
\end{itemize}

\paragraph{Unit relabelings (do not change normalized quantities).}
For any \(\alpha,\beta>0\),
\[
(\tau_{0},\ell_{0})\mapsto(\alpha\tau_{0},\,\beta\ell_{0})\quad\Rightarrow\quad
\frac{\tau_{\mathrm{rec}}}{\tau_{0}}=K,\ \ \frac{\lambda_{\mathrm{kin}}}{\ell_{0}}=K,\ \ \frac{S}{\hbar}=J
\ \ \text{(unchanged)}.
\]
This is the non-circularity already used in the main text.

\subsection*{C.1 Alternate phase and tick conventions}
Some readers prefer different phase-per-cycle or tick-per-cycle conventions. Let
\[
L>0\quad(\text{phase per full cycle}),\qquad M\in\mathbb{N}_{\ge 1}\quad(\text{ticks per cycle}).
\]
The default is \((L,M)=(2\pi,8)\). Replacing \((2\pi,8)\) by \((L,M)\) changes only the proportionality constant:
\[
K_{(L,M)}:=\frac{L}{M\,\ln\varphi},\qquad
\tau_{\mathrm{rec}}=K_{(L,M)}\,\tau_{0},\qquad
\lambda_{\mathrm{kin}}=K_{(L,M)}\,\ell_{0}.
\]
Normalized invariants remain
\[
\frac{\tau_{\mathrm{rec}}}{\tau_{0}}=K_{(L,M)},\qquad
\frac{\lambda_{\mathrm{kin}}}{\ell_{0}}=K_{(L,M)},\qquad
\frac{S}{\hbar}=J.
\]
\emph{Examples.}
\begin{itemize}
  \item \(\pi\)-period convention: \(L=\pi\), \(M=8\) \(\Rightarrow\) \(K=\pi/(8\ln\varphi)\).
  \item Sixteen-tick convention: \(L=2\pi\), \(M=16\) \(\Rightarrow\) \(K=(2\pi)/(16\ln\varphi)=\tfrac12\cdot (2\pi)/(8\ln\varphi)\).
\end{itemize}

\subsection*{C.2 Alternate log bases (purely cosmetic)}
If one writes gaps with base–10 logs, define \(\delta_{\mathrm{gap}}^{(10)}:=\log_{10}\varphi\). Since \(\ln\varphi=(\ln 10)\,\delta_{\mathrm{gap}}^{(10)}\),
\[
K=\frac{2\pi}{8\ln\varphi}
=\frac{2\pi}{8(\ln 10)\,\delta_{\mathrm{gap}}^{(10)}},
\]
and all formulas are unchanged after this substitution. The proofs, which use \(\cosh t\), are already base–free once \(t=\ln x\) is fixed.

\subsection*{C.3 Action normalization and equivalence}
Some texts use a rescaled cost \(J^{\star}=a\,J+b\) with \(a>0\). The bridge enforces \(b=0\) (from \(J(1)=0\)) and \(a=1\) in the default display \(S/\hbar=J\). If a reader insists on \(J^{\star}=a\,J\), the displays remain equivalent by redefining \(\hbar^{\star}:=\hbar/a\):
\[
\frac{S}{\hbar^{\star}}=\frac{S}{\hbar/a}=a\,\frac{S}{\hbar}=a\,J=J^{\star}.
\]
Thus alternative action normalizations amount to a relabeling of \(\hbar\) and do not alter any dimensionless statement or test.

\subsection*{C.4 Changing the parity dimension}
If a different parity dimension \(D\) is adopted (so the minimal cycle is \(M=2^{D}\)), use the rule in C.1 with that \(M\):
\[
K_{(2\pi,\,2^{D})}=\frac{2\pi}{2^{D}\ln\varphi},\qquad
\frac{\tau_{\mathrm{rec}}}{\tau_{0}}=K_{(2\pi,\,2^{D})},\qquad
\frac{\lambda_{\mathrm{kin}}}{\ell_{0}}=K_{(2\pi,\,2^{D})}.
\]
No proof step changes; only the display constant \(K\) is replaced accordingly.

\subsection*{C.5 Summary (translation dictionary)}
Given any alternate conventions \((L,M)\), optional action scale \(a>0\), and optional base–10 gap \(\delta_{\mathrm{gap}}^{(10)}\), the unique translations back to the defaults are:
\[
K\leftarrow \frac{L}{M\,\ln\varphi}
=\frac{L}{M\,(\ln 10)\,\delta_{\mathrm{gap}}^{(10)}},\qquad
\hbar\leftarrow \frac{\hbar^{\star}}{a},\qquad
J\leftarrow \frac{J^{\star}}{a}.
\]
All normalized invariants \(\tau_{\mathrm{rec}}/\tau_{0}\), \(\lambda_{\mathrm{kin}}/\ell_{0}\), and \(S/\hbar\) then coincide with the defaults. No theorem or decision rule changes under these translations.

\section*{Appendix D.\ Upstream Inputs (Exact Statements)}

All items below are taken as proved upstream and are restated here verbatim for convenience. They are purely dimensionless and are not re-proved in this paper.

\begin{theorem}[UP-1: Cost uniqueness and log-axis form]
There exists a unique symmetric multiplicative cost
\[
J:\ \mathbb{R}_{>0}\to\mathbb{R},\qquad
J(x)=\tfrac12\!\left(x+\frac{1}{x}\right)-1,
\]
characterized by \(J(1)=0\) and \(J(x)=J(1/x)\). On the log axis one has
\[
J(e^{t})=\cosh t-1\quad(t\in\mathbb{R}).
\]
In particular, \(J(e^{t})\ge 0\) with equality iff \(t=0\), \(\frac{d}{dt}J(e^{t})\big|_{t=0}=0\), and \(\frac{d^{2}}{dt^{2}}J(e^{t})\big|_{t=0}=1\).
\end{theorem}

\begin{theorem}[UP-2: Quantized tick and discrete potential theory]
Let \(U\) be a set and \(R\subseteq U\times U\) a directed reach relation. There exists a fundamental positive increment \(\delta\) and a potential \(p:U\to\mathbb{Z}\) such that for every edge \((a,b)\in R\),
\[
p(b)-p(a)=\delta.
\]
Consequently, for any chain \(a=u_{0}\leadsto u_{1}\leadsto\cdots\leadsto u_{n}=b\) of length \(n\),
\[
p(b)-p(a)=n\cdot\delta.
\]
Moreover, if \(p,q:U\to\mathbb{Z}\) satisfy the same edge increment \(\delta\), then on each reach component there exists a constant \(c\in\mathbb{Z}\) such that \(p=q+c\) on that component.
\end{theorem}

\begin{theorem}[UP-3: Minimal parity cycle in \(D=3\)]
Let \(\mathrm{Pattern}(3)=\{0,1\}^{\{1,2,3\}}\). There exists a cycle that visits every element of \(\mathrm{Pattern}(3)\) exactly once before repeating, and any such cycle has period exactly
\[
2^{3}=8.
\]
Equivalently, the eight-tick partition in three-bit parity space is both attainable and minimal.
\end{theorem}

\begin{theorem}[UP-4: Constants layer (golden-ratio gap)]
Let \(\varphi=\tfrac{1+\sqrt{5}}{2}\) be the positive solution of \(x=1+\tfrac{1}{x}\) (so \(\varphi^{2}=\varphi+1\)). Define the dimensionless gap
\[
\delta_{\mathrm{gap}}:=\ln\varphi>0.
\]
\end{theorem}
\section*{References}
\begin{thebibliography}{9}
\bibitem{JCGM200} JCGM~200:2012, International vocabulary of metrology (VIM), 3rd edition, Joint Committee for Guides in Metrology.
\bibitem{JCGM100} JCGM~100:2008, Evaluation of measurement data — Guide to the expression of uncertainty in measurement (GUM 1995 with minor corrections), Joint Committee for Guides in Metrology.
\bibitem{SI9} BIPM, The International System of Units (SI), 9th edition, 2019.
\bibitem{Lean4} de Moura L, et al., The Lean~4 theorem prover, 2021--2024, \url{https://leanprover.github.io}.
\bibitem{mathlib} The mathlib community, The Lean mathematical library (mathlib), 2019--2025, \url{https://leanprover-community.github.io/}.
\bibitem{CODATA2018} Mohr P J, Newell D B, Taylor B N and Tiesinga E 2021 CODATA recommended values of the fundamental physical constants: 2018 \textit{Rev. Mod. Phys.} \textbf{93} 025010.
\bibitem{MiP-Second} BIPM, Mise en pratique for the definition of the second (2019, updated), \url{https://www.bipm.org/en/si-second}.
\bibitem{MiP-Metre} BIPM, Mise en pratique for the definition of the metre (2019, updated), \url{https://www.bipm.org/en/si-metre}.

\bibitem{CODATA2022} Tiesinga E, Mohr P J, Newell D B and Taylor B N 2023 CODATA recommended values of the fundamental physical constants: 2022 Preprint arXiv:2302.00881.
\bibitem{NISTSP330} NIST Special Publication 330: The International System of Units (SI), 2019 Ed., NIST.
\bibitem{NISTSP811} NIST Special Publication 811: Guide for the Use of the International System of Units (SI), 2008 Ed., NIST.
\bibitem{Udem2002} Udem T, Holzwarth R and Hänsch T W 2002 Optical frequency metrology Nature 416 233–237.
\bibitem{Ludlow2015} Ludlow A D, Boyd M M, Ye J, Peik E and Schmidt P O 2015 Optical atomic clocks Rev. Mod. Phys. 87 637–701.
\bibitem{Allan1966} Allan D W 1966 Statistics of atomic frequency standards Proc. IEEE 54 221–230.
\bibitem{NISTSP1065} NIST Special Publication 1065: Handbook of Frequency Stability Analysis, 2008, NIST.
\bibitem{CGPM2018} 26th CGPM (2018): Resolution 1—On the revision of the International System of Units (SI).


\bibitem{ISO17025} ISO/IEC~17025:2017, General requirements for the competence of testing and calibration laboratories.
\bibitem{CIPM_MRA} CIPM MRA 1999, Mutual recognition of national measurement standards and of calibration and measurement certificates issued by NMIs.
\bibitem{JCGM101} JCGM~101:2008, Evaluation of measurement data — Supplement~1 to the GUM: Propagation of distributions using a Monte Carlo method.
\bibitem{JCGM102} JCGM~102:2011, Evaluation of measurement data — Supplement~2 to the GUM: Extension to any number of output quantities.
\bibitem{MiP-Kg} BIPM, Mise en pratique for the definition of the kilogram (2019, updated), \url{https://www.bipm.org}.
\bibitem{CundiffYe2003} Cundiff S~T and Ye J 2003 Colloquium: Femtosecond optical frequency combs \textit{Rev. Mod. Phys.} \textbf{75} 325–342.
\bibitem{Diddams2001} Diddams S~A et~al. 2001 An optical clock based on a single trapped $^{199}$Hg$^+$ ion \textit{Science} \textbf{293} 825–828.
\bibitem{Riehle2015} Riehle F 2015 Towards a redefinition of the second based on optical atomic clocks \textit{Comptes Rendus Physique} \textbf{16} 506–515.
\bibitem{Student1908} Student 1908 The probable error of a mean \textit{Biometrika} \textbf{6} 1–25.
\bibitem{Billingsley1995} Billingsley P 1995 \textit{Probability and Measure}, 3rd ed. (Wiley).
\bibitem{Quinn2014G} Quinn T~J, Speake C, Davis R~S, Richman S~J and Berry J~P 2014 The Newtonian constant of gravitation: a constant too difficult to measure? \textit{Phil. Trans. R. Soc. A} \textbf{372} 20140253.
\end{thebibliography}
\end{document}
